\documentclass[12pt]{article}

\usepackage{fullpage}

\begin{document}

\subsection*{Project for the Numerical Course: ATSC409/EOSC511/ATSC506}

The project should be the solution numerically of a interesting
problem.  For graduate students it may be directly related to your
research.  It should not be too complicated (no analysis of full
3-dimensional solutions of the Navier-Stokes equations) nor trivial.

We are happy to help you select a project.  Set-up a time to come and
talk to us.  Undergrads we have a list of recommended/suggested projects.

The first step is a project proposal: see the schedule for due dates. The proposal should include an introduction to the science of the problem, including appropriate references, and what methods you plan to use and why. Two pages long, single typed and upto one figure.

It will be marked on:
\begin{enumerate}
\item The write-up itself, clarity, referencing, completeness, brevity etc
\item Method choice and justification
\item Explaining the science behind your problem.
\end{enumerate}

The finished project is presented two ways:
both as a report, about 10 pages, see marking scheme below and
as a presentation: here the point is to "teach" your peers what you
did.  It is okay to skip some detail for clarity and interest (the detail can go in the written report).

Both the graduate and undergraduate projects are marked based on the
same criteria.  Note however that the graduate project is a much
larger piece of work (worth 50\% of the course) whereas the under-grad
project is aimed at 36 hours of work (worth 30\% of the course).

The projects are marked on the following 5 criteria, each has equal weight.
\begin{enumerate}
\item The write-up itself, clarity, referencing, completeness, brevity
etc
\item Method choice and justification
\item Implementation (making it work)
\item Understanding the numerics  (this could include looking at stability or accuracy or speed of computation or something else)
\item Understanding the science (whatever is
relevant to your problem).
\end{enumerate}

Presentations are marked on:
\begin{enumerate}
\item Form: Introduction, Body, Summary
\item Clarity : Clear slides, nice explanations
\item Content : Actually says something about the science/numerics
\item Interestingness
\end{enumerate}

Presentations are 10 min + 3 min for questions + 2 min change over. There are marks for asking your peers two good questions.

\end{document}

%%% Local Variables:
%%% mode: latex
%%% TeX-master: t
%%% End:
